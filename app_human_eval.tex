
\section{Human Evaluation Details}
\label{app:humaneval}

Two separate annotators ranked the 100
E2E outputs and another two annotators ranked the 100 ViGGO outputs.
The annotators were either undergraduate or PhD students experienced in
NLP research but not involved in the paper. Three were native English 
speakers and the fourth was a highly fluent English speaker.
When computing Kendall's $\tau$ on E2E, three instances were not computable
because one annotator gave all three outputs the same rank. These 
three instances were assigned $\tau=0$ equivalent to no correlation.


Annotators were given the following instructions and then made their 
ranking annotations in Google Sheet:




\paragraph{Instructions:} \textit{You will be shown 3 utterances that are 
informing you about either a restaurant or a video game. 
Please rank the utterances according to their naturalness
(i.e. fluency and/or degree to which you believe they were written by a native English speaker). 1 = most natural, 3 = least natural.}


\textit{Here is an example:}\\




\begin{centering}
\begin{tabular}{p{5.4cm} c}
\toprule
   &  Rank\\
   \midrule
    (0) There is an English food place near the Sorrento with a price range of less than \pounds20 called Blue Spice. & 2 \\
    \midrule
    (1) Blue Spice serves English food for less than \pounds20 and is located near the Sorrento. &  1 \\
    \midrule
    (2) Serving English food near the Sorrento with a price range of less than \pounds20 is Blue Spice.  & 3\\
    \bottomrule
\end{tabular}


\end{centering}

~\\

  \textit{  Here I have decided that (1) feels the most natural, nicely breaking up information into a conjunction, while (2) seems least natural because of its run on gerund phrase in a  copula. (0) is a little bit of a run on but not egregious.}

 \textit{   Do not worry if one utterance does not have all the same or inconsistent facts as the others. Judge them only on their naturalness.}


  \textit{  In many cases you will probably feel that two or more examples are equivalent in naturalness. In this case give them the same rank. E.g.,}


~\\

\begin{centering}

\begin{tabular}{p{5.5cm} c}
\toprule
   &  Rank\\
   \midrule
        (0) There is a place that serves Japanese food in the riverside area near Caf{\'e} Sicilia called the Twenty Two. &   1 \\
   \midrule
        (1) The Twenty Two serves Japanese food and is located near Caf{\'e} Sicilia in the riverside area. &  1 \\
   \midrule
        (2) Serving Japanese food in the riverside area near Caf{\'e} Sicilia is the Twenty Two.  &  2 \\
\bottomrule
\end{tabular}

\end{centering}

~\\


\textit{When making ties, make sure the next lowest rank follows numerically, i.e. if there is a tie for 1, the next lowest rank should be 2. In other words don't do this:}

~\\

\begin{centering}

\begin{tabular}{p{5.5cm} c}
\toprule
   &  Rank\\
   \midrule
            (0) There is a place that serves Japanese food in the riverside area near Caf{\'e} Sicilia called the Twenty Two.  &  1\\
      (1) The Twenty Two serves Japanese food and is located near Caf{\'e} Sicilia in the riverside area. &  1 \\
          (2) Serving Japanese food in the riverside area near Caf{\'e} Sicilia is the Twenty Two.  &  3 \\
\bottomrule
\end{tabular}

\end{centering}

~\\

        \textit{    You will annotate 100 sets of 3 utterances.}


%?
%?

%?
