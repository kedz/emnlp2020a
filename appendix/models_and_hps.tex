\section{Models and Hyper-paramter Search Details}
\label{app:hps}

\subsection{General Details}

Utterance text was sentence and word tokenized, and all tokens were
lower-cased. A special \textit{sentence-boundary} token was inserted between
sentences. All words occurring fewer than 3 times on the training set were
replaced with a special \textit{unknown} token.  %The resulting utterance
%vocabulary sizes are \placeholder{X} and \placeholder{X} for E2E and ViGGO
%datasets respectively.  
We used a batch size of 128 for all biGRU and
Transformer models.  All models were trained on a single Nvidia Tesla v100 for
at most 700 epochs.

\paragraph{Delexicalization} The ViGGO corpus is relatively small and the
attributes \Atr{name}, \Atr{developer}, \Atr{release\_year},
\Atr{expected\_release\_date}, and \Atr{specifier}~can have values that are
only seen several times during training. Neural models often struggle to learn
good representations for infrequent inputs, which can, in turn, lead to poor
test-set generalization. To alleviate this, we delexicalize these values in
the utterance. That is, we replace them with an attribute specific placeholder
token.

\label{app:specifier} Additionally, for \Atr{specifier} whose values come from the open class of
adjectives, we represent the specified adjective with a placeholder which
marks two features, whether it is consonant (C) or vowel initial (V) (e.g.
``\ul{d}ull'' vs. ``\ul{o}ld'') and whether it is in regular (R) or
superlative (S) form (e.g. ``dull'' vs. ``dullest'') since these features can
effect the surrounding context in which the adjective is realized.  See the
following lexicalized/delexicalized examples:
\begin{itemize}
        \item \AV{specifier}{oldest}~-- vowel initial, superlative
\begin{itemize}
    \item \textit{What is the oldest game you've played?}
    \item \textit{What is the SPECIFIER\_V\_S game you've played?}
\end{itemize}
        \item \AV{specifier}{old}~-- vowel initial, regular

\begin{itemize}
    \item \textit{What is an old game you've played?}
    \item \textit{What is an SPECIFIER\_V\_R game you've played?}
\end{itemize}

        \item \AV{specifier}{new}~-- consonant initial, regular

\begin{itemize}
    \item \textit{What is a new game you've played?}
    \item \textit{What is a SPECIFIER\_C\_R game you've played?}
\end{itemize}
\end{itemize}

All generated delexicalized utterances are post-processed with the
corresponding attribute-values before computing evaluation metrics (i.e., 
they are re-lexicalized with the appropriate value strings from the input MR).





\subsection{biGRU Model Definition}
\label{sec:bahdanau}

Let $\Attrs$ be the encoder input vocabulary, and  $\encEmbs \in
\reals^{\size{\Attrs} \times \embDim}$ an associated word embedding matrix
where $\encEmbs_\attr \in \reals^{\embDim}$ denotes the $\embDim$-dimensional
embedding for each $\attr \in \Attrs$. 
Given a linearized MR $\lin(\mr) = \inseq= \left[ \da, \attr_1, \attr_2, \ldots,
\attr_{\size{\mr}}\right] \in \Attrs^{\inSize}$ where the length
of the sequence is $\inSize = \size{\mr} + 1$,
let $\encWordEmb_i = \encEmbs_{\inseq_i}$ for $i \in \{1, \ldots \inSize\}$.

%the associated sequence of encoder input embeddings is



The hidden states of the first GRU encoder layer are
computed as
\begin{align*}
    \fhstate^{(1)}_0 & = \rhstate^{(1)}_{\inSize + 1} = \zeroEmb \\
    \fhstate^{(1)}_i &= \fgru\left(\encWordEmb_i, \fhstate^{(1)}_{i-1};
            \fencgruparams{1}\right) &
    \textrm{for $i \in 1,\ldots,\inSize$}\\
    \rhstate^{(1)}_i &= \fgru\left(\encWordEmb_i, \rhstate^{(1)}_{i+1};
            \rencgruparams{1}\right) &
    \textrm{for $i \in \inSize,\ldots,1$}\\
    \hstate_i^{(1)} & = \left[\fhstate_i, \rhstate_i\right]
\end{align*}
where $\left[\cdot\right]$ is the concatenation operator, $\fhstate^{(1)}_i,\rhstate^{(1)}_i \in \reals^{\hidDim}$, $\hstate^{(1)}_i \in  \reals^{2\hidDim}$,
and $\fencgruparams{1}$ and $\rencgruparams{1}$ are the forward and backward
encoder GRU parameters.

When using a two layer GRU, we similarly compute
\begin{align*}
    \fhstate^{(2)}_0 & =\rhstate^{(2)}_{\inSize+1}  = \zeroEmb \\
    \fhstate^{(2)}_i &= \fgru\left(\hstate^{(1)}_i, \fhstate^{(2)}_{i-1};
            \fencgruparams{2}\right) &
    \textrm{for $i \in 1,\ldots,\inSize$}\\
    \rhstate^{(2)}_i &= \fgru\left(\hstate^{(1)}_i, \rhstate^{(2)}_{i+1};
            \rencgruparams{2}\right) &
    \textrm{for $i \in \inSize,\ldots,1$}\\
    \hstate_i^{(2)} & = \left[\fhstate_i, \rhstate_i\right]
\end{align*}
where $\fhstate^{(2)}_i,\rhstate^{(2)}_i \in \reals^{\hidDim}$, $\hstate^{(2)}_i \in  \reals^{2\hidDim}$,
and $\fencgruparams{2}$ and $\rencgruparams{2}$ are the forward and backward
encoder GRU parameters for the second layer.

Going forward, let $\hstate_i$ correspond to the final encoder output, 
i.e. $\hstate_i = \hstate_i^{(1)}$
in the one-layer biGRU case, and $\hstate_i=\hstate_i^{(2)}$ in the two layer 
case.


Let $\uttVocab$ be the vocabulary of utterance tokens, and 
$\decEmbs \in \reals^{\size{\uttVocab} \times \embDim}$
an associated embedding matrix, where
$\decEmbs_\utttok \in \reals^{\embDim}$ denotes a  $\embDim$-dimensional embedding for
each $\utttok \in \uttVocab$.

%\placeholder{TODO: Make this true for all layers}
Given the decoder input sequence $\utt = \utttok_1, \utttok_2, \ldots, 
\utttok_\size{\utt}$, 
let $\decWordEmb_i = \decEmbs_{\utttok_i}$ for $i \in \{1, \ldots \outSize\}$.
where $\outSize = \size{\utt} - 1$

\begin{table*}
\center
\begin{tabular}{clccc ccc ccc}
\toprule
&Model & Layers & LS & WD & Optim. & LR & Emb. & Attention & Params & Train Time \\
\midrule
    \parbox[t]{2mm}{\multirow{6}{*}{\rotatebox[origin=c]{90}{E2E}}} 
 & \lsshort{Rnd} & 2 & 0.1 & $10^{-5}$ & Adam & $10^{-5}$ & untied & Bahdanau & 14,820,419 & 31.16 \\
 & \lsshort{Fp} & 2 & 0.1 & $10^{-5}$ & SGD & $0.1$ & untied & Bahdanau & 14,820,003 & 24.78\\ 
 & \lsshort{If} & 2 & 0.1 & $0.0$ & SGD & $0.5$ & untied & General & 14,557,763 & 26.44 \\
 & \lsshort{If+p} & 2 & 0.1 & $0.0$ & SGD & $0.5$ & untied & General & 14,557,763 & 15.23 \\
 & \lsshort{At} & 2 & 0.1 & $10^{-5}$ & Adam & $10^{-5}$ & untied & Bahdanau & 14,820,419 & 26.07 \\
 & \lsshort{At+p} & 2 & 0.1 & $10^{-5}$ & Adam & $10^{-5}$ & untied & Bahdanau & 14,820,419 & 36.49 \\
\midrule
    \parbox[t]{2mm}{\multirow{6}{*}{\rotatebox[origin=c]{90}{ViGGO}}} 
 & \lsshort{Rnd} & 2 & 0.1 & $10^{-5}$ & SGD & $0.25$ & untied & General & 14,274,865 & 20.69 \\
 & \lsshort{Fp} & 1 & 0.1 & $10^{-5}$ & Adam & $10^{-5}$ & untied & Bahdanau & 7,718,193 & 30.07\\ 
 & \lsshort{If} & 1 & 0.0 & $0.0$ & SGD & $0.5$ & untied & Bahdanau & 7,712,049 & 11.62\\ 
 & \lsshort{If+} & 1 & 0.0 & $0.0$ & SGD & $0.5$ & untied & Bahdanau & 7,712,049 & 5.08\\ 
 & \lsshort{At} & 2 & 0.1 & $0.0$ & Adam & $10^{-5}$ & untied & Bahdanau & 14,537,521 & 21.01\\ 
 & \lsshort{At+p} & 2 & 0.1 & $0.0$ & Adam & $10^{-5}$ & untied & Bahdanau & 14,537,521 & 14.95\\ 
\bottomrule
\end{tabular}

\caption{Winning hyperparameter settings for biGRU models. LS and WD indicate label smoothing and weight decay respectively. Train time is in hours.}
\label{tab:gruparams}
\end{table*}


We compute the hidden states of the $i$-th layer of the decoder as,
\begin{align*}
\gstate^{(i)}_0 & = \tanh\left(\weight{i} \hstate^{(i)}_\inSize + \bias{i}\right)\\
\end{align*}
\noindent for $j \in 1,\ldots,\outSize$
\begin{align*}
    \gstate^{(i)}_j &= \fgru\left(\gstate^{(i-1)}_j, \gstate^{(i)}_{j-1};
            \decgruparams{i}\right) \\
\end{align*}
where $\gstate^{(0)}_j = \decWordEmb_j$, $\gstate^{(i)}_j \in \reals^{\hidDim}$, $\weight{i} \in \reals^{\hidDim \times 2\hidDim}$, $\bias{i} \in \reals^{\hidDim}$ and $\decgruparams{i}$ are the decoder GRU parameters.

Going forward, let $\gstate_i$ correspond to the final decoder output, 
i.e. $\gstate_i = \gstate_i^{(1)}$
in the one-layer biGRU case, and $\gstate_i=\gstate_i^{(2)}$ in the two layer 
case.



Then the decoder states attend to the encoder states,
\begin{align*}
    \astate_i & = \sum^{\inSize}_{j=1}\alpha_{i,j}\hstate_j & \textrm{for $i \in 1,\ldots, \outSize$}
\end{align*}
where $\alpha_{i,j} \in (0, 1)$ is the 
attention weight of decoder state $i$ on encoder state $j$ and 
$\sum^{\inSize}_{j=1}\alpha_{i,j} = 1$.
We compute 
attention in one of two ways (the attention method is a hyperparemeter
option):
\begin{enumerate}
\item Feed-forward ``Bahdanau'' style attention \cite{bahdanau2015},
    also known as ``concat'' \cite{luong2015}: 
\[ \alpha_{i,j} = \attnvec \tanh\left(\attnweight\left[\begin{array}{c} 
    \gstate_i \\ \hstate_j\end{array}\right] \right)  \] with 
$\attnweight \in \reals^{\hidDim \times 3\hidDim}$ and 
$\attnvec \in \reals^{\hidDim}$.
\item ``general'' \cite{luong2015} :
    \[ \alpha_{i,j} = \gstate_i \attnweight  \hstate_j  \] with $\attnweight \in \reals^{\hidDim \times 2\hidDim}$.
\end{enumerate}

Finally, for $i \in 1, \ldots, \outSize$ we compute
\begin{align*}
\zstate_i =  \tanh\left(\weight{z}\left[\begin{array}{l} 
    \gstate_i\\ \astate_i \end{array}\right] + \bias{z}\right) \\
\end{align*} and \\

\noindent $p\left(\utttok_{i+1}|\utttok_{\le i}, \lin(\mr)\right)  =$
\begin{align*}
    \softmax\left(\weight{o}\zstate_i + \bias{o}\right)_{\utttok_{i+1}}  
\end{align*}
where $\weight{z} \in \mathbb{R}^{\hidDim \times 3\hidDim}$, $\bias{z} \in\reals^{\hidDim}$, 
$\bias{o} \in \reals^{\size{\uttVocab}}$,
and $\weight{o} \in \reals^{\size{\uttVocab} \times \hidDim}$ is the output
    embedding matrix. 
    As a hyperparamter setting, we consider tieing the decoder 
    input and output
    embedding matrices, i.e. $\decEmbs = \weight{o}$.
    Dropout of 0.1 is applied to all embedding, GRU outputs, and linear
    layer outputs. We set $\embDim = \hidDim = 512$.


\subsection{biGRU Hyperparameter Search}

We grid-search over the following hyperparameter values:

\begin{itemize}
    \item \textbf{Layers:}  $1$, $2$ 
    \item \textbf{Label Smoothing:} $0$, $0.1$
    \item \textbf{Weight Decay:} $0$, $10^{-5}$
    \item \textbf{Optimizer/Learning Rate:} Adam/$10^{-3}$, Adam/$10^{-4}$, Adam/$10^{-5}$,
        SGD/$0.5$, SGD/$0.25$, SGD/$0.1$
    \item \textbf{Tie Decoder Embeddings:} tied, untied
     \item \textbf{Attention:} Bahdanau, General
       \end{itemize}

During hyperparameter search, we train for at most 500
epochs, evaluating \textsc{Bleu} every 25 epochs to select the best model.
We decay the learning if validation log-likelihood stops increasing 
for five epochs. We decay the learning rate by $lr^{i+1}= .99 \times lr^{i}$.

Winning hyperparameter settings are presented \autoref{tab:gruparams}.


%\begin{itemize}
%    \item E2E
%\begin{itemize}
%    \item \lsshort{Rnd} --- 2 layers, $0.1$ label smoothing, $10^{-5}$
%        weight decay, Adam with lr $=10^{-5}$, untied embeddings, Bahdanau attention
%        \begin{itemize}
%            \item Params: 14,820,419
%            \item Avg. Train Time: 31.16 hours
%        \end{itemize}
%    \item \lsshort{Fp} --- 2 layers, $0.1$ label smoothing, $10^{-5}$
%        weight decay, SGD with lr $=0.1$, untied embeddings, Bahdanau attention
%        \begin{itemize}
%            \item Params: 14,824,003
%            \item Avg. Train Time: 24.78 hours
%        \end{itemize}
%    \item \lsshort{If} --- 2 layers, $0.1$ label smoothing, $0.0$
%        weight decay, SGD with lr $=0.5$, untied embeddings, General attention
%        \begin{itemize}
%            \item Params: 14,557,763
%            \item Avg. Train Time: 26.44 hours
%            \item Avg. Train Time (\textsc{+p}): 15.23 hours
%        \end{itemize}
%    \item \lsshort{At} --- 2 layers, $0.1$ label smoothing, $10^{-5}$
%        weight decay, Adam with lr $=10^{-5}$, untied embeddings, Bahdanau attention
%        \begin{itemize}
%            \item Params: 14,820,419 
%            \item Avg. Train Time: 26.07 hours 
%            \item Avg. Train Time (\textsc{+p}): 36.49 hours 
%        \end{itemize}
%\end{itemize}
%\item ViGGO
%\begin{itemize}
%    \item \lsshort{Rnd} --- 2 layers, $0.1$ label smoothing, $10^{-5}$
%        weight decay, SGD with lr $=0.25$, untied embeddings, General attention
%        \begin{itemize}
%            \item Params: 14,274,865
%            \item Avg. Train Time: 20.69 hours
%        \end{itemize}
%    \item \lsshort{Fp} --- 1 layers, $0.1$ label smoothing, $10^{-5}$
%        weight decay, Adam with lr $=10^{-5}$, untied embeddings, Bahdanau attention
%        \begin{itemize}
%            \item Params: 7,718,193
%            \item Avg. Train Time: 30.07 hours
%        \end{itemize}
%    \item \lsshort{If} --- 1 layers, $0.0$ label smoothing, $0.0$
%        weight decay, SGD with lr $=0.5$, untied embeddings, Bahdanau attention
%        \begin{itemize}
%            \item Params: 7,712,049
%            \item Avg. Train Time: 11.62 hours
%            \item Avg. Train Time (\textsc{+p}): 5.08 hours
%        \end{itemize}
%    \item \lsshort{At} --- 2 layers, $0.1$ label smoothing, $0.0$
%        weight decay, Adam with lr $=10^{-5}$, untied embeddings, Bahdanau attention
%        \begin{itemize}
%            \item Params: 14,537,521
%            \item Avg. Train Time: 21.01 hours
%            \item Avg. Train Time (\textsc{+p}): 14.95 hours
%        \end{itemize}
%\end{itemize}
%\end{itemize}




\subsection{Transformer Model Definition}

Each Transformer layer is divided into blocks which each have three
parts, (i) layer norm, (ii) feed-forward/attention, and  (iii) skip-connection.
We first define the components used in the transformer blocks before
describing the overall S2S transformer. 
Starting with layer norm \cite{ba2016}, let $\encInput \in \reals^{m\times n}$, then we have
$\layerNorm : \reals^{m \times n} \rightarrow \reals^{m \times n}$,
\[\layerNorm(\encInput; \lnweightv, \lnbias) = \lnweight \odot (\encInput - \boldsymbol{\mu}) \odot \Lambda + \mathbf{b} \]

where $\lnweightv, \lnbias \in \reals^n$ are learned parameters, $\odot$ is the elementwise product, $\lnweight = \left[\lnweightv,\ldots,\lnweightv\right] \in \reals^{m\times n}$ is a tiling of the parameter vector, $\lnweightv$, $m$ times, and  $\boldsymbol{\mu}, \boldsymbol{\Lambda} \in \reals^{m\times n}$ are
defined elementwise as
\[\boldsymbol{\mu}_{i,j} = \frac{1}{n} \sum_{k=1}^n \encInput_{i,k}\]
and 
\[\boldsymbol{\Lambda}_{i,j} = \left(
    \sqrt{ \frac{1}{n-1} \sum_{k=1}^n \left( 
\encInput_{i,k} - \boldsymbol{\mu}_{i,j} \right)^2  + \epsilon}\right)^{-1}\]
respectively. The $\epsilon$ term is a small constant for numerical stability,
set to $10^{-5}$.

The inplace feed-forward layer, $\feedforward$, is a simple single-layer perceptron
with $\relu$ activation 
($\relu(\encInput) = \max\left(\zeroEmb, \encInput\right)$) \cite{nair2010}, applied to each row of an $m \times n$ input matrix, i.e. a sequence of $m$ objects
with $n$ features,\\


\noindent $\feedforward\left(\encInput;\weight{i},\weight{j},\bias{i},\bias{j}\right) =$
\[  \relu\left(\encInput\weight{i} + \bias{i}\right)\weight{j} + \bias{j}     \]
where $\weight{i} \in \reals^{\embDim \times \hidDim}$, $\bias{i} \in \reals^{\hidDim}$,
$\weight{j} \in \reals^{\hidDim \times \embDim}$, $\bias{j} \in \reals^{\embDim}$ are learned parameters and 
matrix-vector additions (i.e. $\mathbf{X} + \mathbf{b}$) are broadcast across
the matrix rows.


The final component to be defined is the multi-head attention, $\MultiAttn$ which is defined
as\\

\noindent  $\MultiAttn(\Query, \Key; \weight{a_1}, \weight{a_2}) =$
\[ \left[ \begin{array}{c} 
            \Attn\left(\Query\weight{a_1}_{1,1}, \Key\weight{a_1}_{2,1}, \Key \weight{a_1}_{3,1} \right),\\
    \vdots \\
            \Attn\left(\Query\weight{a_1}_{1,H}, \Key\weight{a_1}_{2,H}, \Key \weight{a_1}_{3,H} \right)
\end{array} \right] \weight{a_2}
\]

where $\left[\cdot \right]$ indicates column-wise concatenation,  $\weight{a_1}_{1,*} \in \reals^{\embDim\times \embDim / H}$
and $\weight{a_2} \in \reals^{\embDim\times \embDim}$ are learned parameters,
$H$ is the number of attention heads, and  
$\Attn$ is defined,

\[\Attn\left(\mathbf{Q}, \mathbf{K}, \mathbf{V}\right) = \softmax\left(\frac{\mathbf{Q} \mathbf{K}^T}{\sqrt{\embDim}} \right)\mathbf{V}. \]

Additionally, there is a masked variant of attention, $\MultiAttn_{M}$
where the attention is computed 

\[\Attn\left(\mathbf{Q}, \mathbf{K}, \mathbf{V}\right) = \softmax\left(\frac{\mathbf{Q} \mathbf{K}^T \odot \Mask }{\sqrt{\embDim}} \right)\mathbf{V} \]
where $\Mask \in \reals^{\outSize \times \inSize}$ 
is a lower triangular matrix, i.e. values on or below the diagonal are 1
and all other values are $-\infty$. 

Given these definitions, we now define the S2S transformer.
Let $\Attrs$ be the encoder input vocabulary, and  $\encEmbs \in
\reals^{\size{\Attrs} \times \embDim}$ an associated word embedding matrix
where $\encEmbs_\attr \in \reals^{\embDim}$ denotes the $\embDim$-dimensional
embedding for each $\attr \in \Attrs$. 
Given a linearized MR $\lin(\mr) = \inseq= \left[ \da, \attr_1, \attr_2, \ldots,
\attr_{\size{\mr}}\right] \in \Attrs^{\inSize}$ where the length
of the sequence is $\inSize = \size{\mr} + 1$,
let $\encWordEmb_i = \encEmbs_{\inseq_i}$ for $i \in \{1, \ldots \inSize\}$.

Additionally let $\posEmb \in \reals^{\inSize_{max} \times \embDim}$ be a sinusoidal position embedding matrix
defined elementwise with 
\begin{align*}
    \posEmb_{i,2j} & = \sin\left(\frac{i}{10,000^{ \frac{2j}{\embDim} }}\right) \\
    \posEmb_{i,2j+1} & = \cos\left(\frac{i}{10,000^{ \frac{2j}{\embDim} }}\right). 
\end{align*}
The encoder input sequence $\encInput^{(0)} \in \reals^{\inSize \times \embDim}$ is then defined by
\[\encInput^{(0)} = \left[\begin{array}{c} 
            \encWordEmb_1 + \posEmb_1,\\
            \encWordEmb_2 + \posEmb_2,\\
            \vdots \\
        \encWordEmb_\inSize + \posEmb_\inSize
    \end{array}
                        \right] \]

 A sequence of $l$ transformer encoder layers are then applied to the encoder
 input, i.e. $\encInput^{(i+1)} = \operatorname{TF}^{(i)}_{enc}\left(\encInput^{(i)}\right)$.
Each encoder transformer layer computes the following, \\

\noindent \textit{(Self-Attention Block)}\\
\[\tfeA = \layerNorm\left(\encInput^{(i)}; \lnweightv^{(i,1)}, \lnbias^{(i,1)}\right)\]
\[\tfeB = \MultiAttn\left(\tfeA, \tfeA; \weight{i,a_1},  \weight{i,a_2}\right)\]
\[\tfeC = \encInput^{(i)} + \tfeB\]

\noindent \textit{(Feed-Forward Block)}\\
\[\tfeD = \layerNorm\left(\tfeC; \lnweightv^{(i,2)}, \lnbias^{(i,2)}\right)\]
\[\tfeE = \feedforward\left(\tfeD;\weight{i,1},\weight{i,2},\bias{i,1},\bias{i,2}\right)\]
\[ \encInput^{(i+1)} = \tfeC + \tfeE \]


We denote the final encoder output for $l$ layers as $\encInput = \encInput^{(l)}$.  

Let $\uttVocab$ be the vocabulary of utterance tokens, and 
$\decEmbs \in \reals^{\size{\uttVocab} \times \embDim}$
an associated embedding matrix, where
$\decEmbs_\utttok \in \reals^{\embDim}$ denotes a  $\embDim$-dimensional embedding for
each $\utttok \in \uttVocab$.

%\placeholder{TODO: Make this true for all layers}
Given the decoder input sequence $\utt = \utttok_1, \utttok_2, \ldots, 
\utttok_\size{\utt}$, 
let $\decWordEmb_i = \decEmbs_{\utttok_i}$ for $i \in \{1, \ldots \outSize\}$.
where $\outSize = \size{\utt} - 1$

\[\decInput^{(0)} = \left[\begin{array}{c} 
            \decWordEmb_1 + \posEmb_1,\\
            \decWordEmb_2 + \posEmb_2,\\
            \vdots \\
        \decWordEmb_\outSize + \posEmb_\outSize
    \end{array}
                        \right]. \]


A sequence of $l$ transformer decoder layers are then applied to the decoder
 input, i.e. $\decInput^{(i+1)} = \operatorname{TF}^{(i)}_{dec}\left(\decInput^{(i)}\right)$.
Each decoder transformer layer computes the following, \\

~\\~\\ ~\\

\noindent \textit{(Masked Self-Attention Block)}\\
\[\tfdA = \layerNorm\left(\decInput^{(i)}; \lnweightv^{(i,1)}, \lnbias^{(i,1)}\right)\]
\[\tfdB = \MultiAttn_M\left(\tfdA, \tfdA; \weight{i,a_1}, \weight{i,a_2} \right)\]
\[\tfdC = \decInput^{(i)} + \tfdB\]

\noindent \textit{(Encoder-Attention Block)}\\
\[\tfdD = \layerNorm\left(\tfdC; \lnweightv^{(i,2)}, \lnbias^{(i,2)}\right)\]
\[\tfdE = \MultiAttn\left(\tfdD, \encInput; \weight{i,a_3}, \weight{i,a_4}\right)\]
\[\tfdF = \tfdC + \tfdE\]

\noindent \textit{(Feed-Forward Block)}\\
\[\tfdG = \layerNorm\left(\tfdF; \lnweightv^{(i,3)}, \lnbias^{(i,3)}\right)\]
\[\tfdH = \feedforward\left(\tfdG;\weight{i,1},\weight{i,2},\bias{i,1},\bias{i,2}\right)\]
\[ \decInput^{(i+1)} = \tfdF + \tfdH \]

Let the $\decInput = \decInput^{(l)}$ denote the final decoder output,
and let $\decInputi$ be the $i$-th row of $\decInput$ corresponding
to the decoder representation of the $i$-th decoder state. The probability of 
the next word is\\

\noindent $\model\left(\utttok_{i+1}|\utttok_{\le i},\lin(\mr)\right)$
\[  = \softmax\left( \weight{o}\decInputi + \bias{o} \right)_{\utttok_{i+1}} \]
where $\weight{o} \in \reals^{\size{\uttVocab} \times \embDim}$ 
and $\bias{o} \in \reals^{\embDim}$ are learned parameters. 


%We used the Transformer S2S as implemented in 
%\href{https://pytorch.org/}{PyTorch}.
The input embedding dimension is $\embDim= 512$ and inner hidden layer size 
is $\hidDim=2048$. The encoder and decoder have separate parameters.
We used $H=8$ heads in all multi-head attention layers. 
 We used Adam with the learning
rate schedule provided in  \citet{rush2018} (factor=1, warmup=8000).
Dropout was set to 0.1 was applied to input embeddings and each skip 
connection (i.e. the third line in each block definition). As a 
hyperparameter, we optionally tie the decoder input and output embeddings,
i.e. $\decEmbs = \weight{o}$.




%?$(\mu_\encInput)_i = \frac{1}{F} \sum_{j=1}^F \encInput_{i,j} $
%?
%?
%?$(\sigma_\encInput)_{i,j} = \sqrt{\frac{1}{F-1} \sum_{j^\prime=1}^F \left( \encInput_{i,j^\prime} - (\mu_\encInput)_{i,j}\right)^2} $
%?
%?
%?
%?
%?
%?
%?
%?Let $\lin\left(\mr\right) = \left[\da, \attr_1, \ldots, \attr_\mrSize \right]$.
%?The linearized MR is first embedded both with word and position
%?beddings which are summed to compute the first encoder layer input,
%?\[ \encInput^{(0)} = \left[\begin{array}{c}
%?        \mrEmb_\da + \posEmb_\da, \\
%?        \mrEmb_{\attr_1} + \posEmb_{\attr_1},\\
%?        \vdots\\
%?    \mrEmb_{\attr_\mrSize} + \posEmb_{\attr_\mrSize}\end{array} \right] \]
%?
%?We then apply $l$ Transformer encoder layers, 
%?\[ \encInput^{(i)} = \operatorname{TF}_e\left(\encInput^{(i-1)}\right)\quad  \forall i: i \in \{1,\ldots,l\}  \]
%?where,
%?\begin{align*}
%?    \boldsymbol{\check{\encInput}}^{(i-1)} &= \operatorname{LayerNorm}\left(\encInput^{(i-1)}\right)\\
%?    \boldsymbol{\bar{\encInput}}^{(i-1)} &= \operatorname{MultiHeadAttn}\left(\encInput^{(i-1)},\encInput^{(i-1)}\right)\\
%?    \boldsymbol{\hat{\encInput}}^{(i-1)} &= \encInput^{(i-1)} + \boldsymbol{\bar{\encInput}}^{(i-1)}\\
%?    \boldsymbol{\tilde{\encInput}}^{(i-1)} &= \operatorname{LayerNorm}\left(
%?        \boldsymbol{\hat{\encInput}}^{(i-1)} \right)\\
%?    \boldsymbol{\acute{\encInput}}^{(i-1)} &= \operatorname{FeedForward}\left(
%?        \boldsymbol{\tilde{\encInput}}^{(i-1)} \right)\\
%?\encInput^{(i)} & =  \boldsymbol{\hat{\encInput}}^{(i-1)} +
%?  \boldsymbol{\acute{\encInput}}^{(i-1)} 
%?\end{align*}
%?
%?We indicate the final Transformer encoder output layer with $\encInput = \encInput^{(l)}$.
%?
%?
%?\[ \decInput^{(0)} = \left[\begin{array}{c}
%?        \uttEmb_{\attr_1} + \posEmb_{1},\\
%?        \vdots\\
%?    \mrEmb_{\attr_\mrSize} + \posEmb_{\mrSize}\end{array} \right] \]
%?
%?
%?\begin{align*}
%?    \boldsymbol{\check{\decInput}}^{(i-1)} &= \layerNorm\left(
%?        \decInput^{(i-1)}\right)\\
%?    \boldsymbol{\bar{\decInput}}^{(i-1)} &= \mmhAttn\left(
%?        \decInput^{(i-1)},\decInput^{(i-1)}\right)\\
%?    \boldsymbol{\hat{\decInput}}^{(i-1)} &= 
%?        \decInput^{(i-1)} + \boldsymbol{\bar{\decInput}}^{(i-1)}\\
%?%
%?    \boldsymbol{\check{\decInput}}^{(i-1)} &= \layerNorm\left(
%?        \decInput^{(i-1)}\right)\\
%?    \boldsymbol{\bar{\decInput}}^{(i-1)} &= \mhAttn\left(
%?        \decInput^{(i-1)},\encInput^{(i-1)}\right)\\
%?    \boldsymbol{\hat{\decInput}}^{(i-1)} &= 
%?        \decInput^{(i-1)} + \boldsymbol{\bar{\decInput}}^{(i-1)}\\
%?%
%?    \boldsymbol{\tilde{\decInput}}^{(i-1)} &= \operatorname{LayerNorm}\left(
%?        \boldsymbol{\hat{\decInput}}^{(i-1)} \right)\\
%?    \boldsymbol{\acute{\decInput}}^{(i-1)} &= \operatorname{FeedForward}\left(
%?        \boldsymbol{\tilde{\decInput}}^{(i-1)} \right)\\
%?\decInput^{(i)} & =  \boldsymbol{\hat{\decInput}}^{(i-1)} +
%?  \boldsymbol{\acute{\decInput}}^{(i-1)} 
%?\end{align*}
%?
%?
%?\[\Attn\left(\mathbf{Q}, \mathbf{K}, \mathbf{V}\right) = \softmax\left(\frac{\mathbf{Q} \mathbf{K}^T}{\sqrt{d_k}} \right)\mathbf{V} \]
%?
%?
%?\noindent  $\mhAttn(\Query, \Key) =$
%?\[ \left[ \begin{array}{c} 
%?    \Attn\left(\Query\mathbf{W}^{(Q)}_1, \Key\mathbf{W}^{(K)}_1, \Key \mathbf{W}^{(V)}_1 \right),\\
%?    \vdots \\
%?    \Attn\left(\Query\mathbf{W}^{(Q)}_H, \Key\mathbf{W}^{(K)}_H, \Key \mathbf{W}^{(V)}_H \right),
%?    \end{array} \right] \mathbf{W}^{(O)}
%?\]
%?
%?
%?\[\feedforward\left(\encInput \right) = \max\left(0, \encInput\mathbf{W} + \mathbf{b} \right)\mathbf{W} + \mathbf{b}\]
%?
%?\[ \model\left(\utt_i|\utt_{<i},\lin(\mr)\right) = \softmax\left( \mathbf{W}\decInputi + \mathbf{b} \right) \]
%?
%?
%?
%?\[\layerNorm(\encInput) = \mathbf{A} \odot (\encInput - \mu_\encInput) \odot \left(\sigma_\encInput + \epsilon\right)^{-1} + \mathbf{b} \]
%?
%?$(\mu_\encInput)_i = \frac{1}{F} \sum_{j=1}^F \encInput_{i,j} $
%?
%?
%?$(\sigma_\encInput)_{i,j} = \sqrt{\frac{1}{F-1} \sum_{j^\prime=1}^F \left( \encInput_{i,j^\prime} - (\mu_\encInput)_{i,j}\right)^2} $
%?
%Let $X = [v_1, \ldots, v_{\size{\mr}}] \in \mathbb{R}^{\size{\mr} \times 512 }$, i.e. the concatenation of the embeddings of the encoder input
%
%
%\[
%
%    H = \operatorname{Multi-Head Attention}(X, X, X)W 
%
%\]


\begin{table*}
\center
\begin{tabular}{clccc ccc}
\toprule
&Model & Layers & LS & Emb. & Params & Train Time \\
\midrule
    \parbox[t]{2mm}{\multirow{6}{*}{\rotatebox[origin=c]{90}{E2E}}} 
 & \lsshort{Rnd} & 1 & 0.1 & tied & 7,966,787 & 18.09 \\
 & \lsshort{Fp} & 1 & 0.1 & tied & 7,970,371 & 17.30 \\
 & \lsshort{If} & 1 & 0.1 & untied & 8,525,379 & 17.52 \\
 & \lsshort{If+p} & 1 & 0.1 & untied & 8,525,379 & 28.11 \\
 & \lsshort{At} & 2 & 0.1 & untied & 15,881,795 & 23.73 \\
 & \lsshort{At+p} & 2 & 0.1 & untied & 15,881,795 & 29.39 \\
\midrule
    \parbox[t]{2mm}{\multirow{6}{*}{\rotatebox[origin=c]{90}{ViGGO}}} 
 & \lsshort{Rnd} & 2 & 0.0 & untied & 15,598,897 & 11.22 \\
 & \lsshort{Fp} & 2 & 0.1 & untied & 15,605,041 & 9.68 \\
 & \lsshort{If} & 2 & 0.1 & untied & 15,598,897 & 11.35 \\
 & \lsshort{If+p} & 2 & 0.1 & untied & 15,598,897 & 9.09 \\
 & \lsshort{At} & 2 & 0.1 & untied & 15,598,897 &7.26 \\
 & \lsshort{At+p} & 2 & 0.1 & untied & 15,598,897 &5.87 \\
\bottomrule
\end{tabular}

\caption{Winning hyperparameter settings for Transformer models (trained from scratch). LS indicates label smoothing. Train time is in hours.}
\label{tab:tfparams}
\end{table*}


\subsection{Transformer Hyperparameter Search}
We grid searched over the following Transformer hyper-parameters:
\begin{itemize}
    \item \textbf{Tied Decoder Embeddings:} tied, untied
    \item \textbf{Layers:} 1, 2
    \item \textbf{Label Smoothing:} 0.0, 0.1
\end{itemize}

During hyperparameter search, we train for at most 500
epochs, evaluating \textsc{Bleu} every 25 epochs to select the best model.
%We now list the winning hyper-parameter settings:

Winning hyperparameter settings are presented \autoref{tab:tfparams}.


%\begin{itemize}
%    \item E2E
%\begin{itemize}
%    \item \lsshort{Rnd} --- 1 layer, $0.1$ label smoothing, tied embeddings
%        \begin{itemize}
%            \item Params: 7,966,787
%            \item Avg. Train Time: 18.09 hours
%        \end{itemize}
%    \item \lsshort{Fp} --- 1 layers, $0.1$ label smoothing, tied embeddings
%        \begin{itemize}
%            \item Params: 7,970,371
%            \item Avg. Train Time: 17.30 hours
%        \end{itemize}
%    \item \lsshort{If} --- 1 layers, $0.1$ label smoothing, untied embeddings
%        \begin{itemize}
%            \item Params: 8,525,379
%            \item Avg. Train Time: 17.52 hours
%            \item Avg. Train Time (\textsc{+p}): 28.11 hours
%        \end{itemize}
%    \item \lsshort{At} --- 2 layers, $0.1$ label smoothing, untied embeddings
%        \begin{itemize}
%            \item Params: 15,881,795
%            \item Avg. Train Time: 23.73 hours
%            \item Avg. Train Time (\textsc{+p}): 29.39 hours
%        \end{itemize}
%\end{itemize}
%\item ViGGO
%\begin{itemize}
%    \item \lsshort{Rnd} --- 2 layers, $0.0$ label smoothing, untied embeddings
%        \begin{itemize}
%            \item Params: 15,598,897
%            \item Avg. Train Time: 11.22 hours
%        \end{itemize}
%    \item \lsshort{Fp} --- 2 layers, $0.1$ label smoothing, untied embeddings
%        \begin{itemize}
%            \item Params: 15,605,041
%            \item Avg. Train Time: 9.68 hours
%        \end{itemize}
%    \item \lsshort{If} --- 2 layers, $0.1$ label smoothing, untied embeddings
%        \begin{itemize}
%            \item Params: 15,598,897
%            \item Avg. Train Time: 11.35 hours
%            \item Avg. Train Time (\textsc{+p}): 9.09 hours
%        \end{itemize}
%    \item \lsshort{At} --- 2 layers, $0.1$ label smoothing, untied embeddings
%        \begin{itemize}
%            \item Params: 15,598,897
%            \item Avg. Train Time: 7.26 hours
%            \item Avg. Train Time (\textsc{+p}): 5.87 hours
%        \end{itemize}
%\end{itemize}
%\end{itemize}



\subsection{BART Model Hyperparameters}

We use the same settings as the fine-tuning for the CNN-DailyMail 
summarization task,
although we modify the maximum number of updates to be roughly to be 
equivalent to 10 epochs on the training set when using a 500 token batch
size, since 
the number of updates effects
the learning rate scheduler. We selected the model iterate with 
lowest validation set cross-entropy. 

While BART is unlikely to have seen any linearized MR in its pretraining
data, its use of sub-word encoding  allows it to encode
arbitrary strings. Rather than extending it's encoder input vocabulary to
add the MR tokens, we simply format the input MR as a string 
(in the correpsonding linearization order), e.g. ``inform rating=good name=NAME platforms=PC platforms=Xbox''.

\clearpage

\begin{table}[t!]
    \centering


\begin{tabular}{ll cccc}
\toprule
\multicolumn{2}{c}{Model}  &    B$\uparrow$ & R$\uparrow$ & SER$\downarrow$ & OA$\uparrow$    \\
\midrule
\parbox[t]{2mm}{\multirow{8}{*}{\rotatebox[origin=c]{90}{biGRU}}}
 & \textsc{\small{Rnd}} &  47.8&59.5&1.84&--- \\
 & \textsc{\small{Fp}} &  49.1&60.3&1.90&--- \\
 & \textsc{\small{If}} &  48.4&59.9&3.86&--- \\
 & \textsc{\small{If +p}} &  48.1&60.0&1.12&--- \\
 & \textsc{\small{At +BgDP}} &  44.2&57.4&0.14&99.0 \\
 & \textsc{\small{At +NDP}} &  48.8&61.2&0.02&99.8 \\
 & \textsc{\small{At +NDP +p}} &  48.7&61.1&0.02&99.8 \\
 & \textsc{\small{At Oracle}} &  56.9&73.2&0.18&99.2 \\
\midrule
\parbox[t]{2mm}{\multirow{8}{*}{\rotatebox[origin=c]{90}{Transformer}}}
 & \textsc{\small{Rnd}} &  47.8&59.4&2.26&--- \\
 & \textsc{\small{Fp}} &  49.2&60.6&2.84&--- \\
 & \textsc{\small{If}} &  49.7&60.9&1.46&--- \\
 & \textsc{\small{If +p}} &  48.9&60.7&1.00&--- \\
 & \textsc{\small{At +BgDP}} &  44.9&57.4&0.28&98.3 \\
 & \textsc{\small{At +NDP}} &  49.3&61.3&0.10&99.4 \\
 & \textsc{\small{At +NDP +p}} &  48.8&61.1&0.08&99.6 \\
 & \textsc{\small{At Oracle}} &  56.9&73.1&0.48&97.5 \\
\midrule
\parbox[t]{2mm}{\multirow{8}{*}{\rotatebox[origin=c]{90}{BART}}}
 & \textsc{\small{Rnd}} &  46.9&59.5&0.48&--- \\
 & \textsc{\small{Fp}} &  48.6&60.3&0.04&--- \\
 & \textsc{\small{If}} &  48.2&60.3&0.18&--- \\
 & \textsc{\small{If +p}} &  47.8&60.1&0.46&--- \\
 & \textsc{\small{At +BgDP}} &  45.9&57.3&0.00&99.9 \\
 & \textsc{\small{At +NDP}} &  49.0&61.2&0.02&99.8 \\
 & \textsc{\small{At +NDP +p}} &  48.8&61.1&0.02&99.8 \\
 & \textsc{\small{At Oracle}} &  55.7&72.8&0.20&99.0 \\
\bottomrule
\end{tabular}


    \caption{E2E validation set (B) \textsc{Bleu}, (R) \textsc{Rouge-L},
    SER, and OA.}
    \label{tab:app.e2e.valid.agg}
\end{table}


\begin{table}[t!]
\centering

\begin{tabular}{ll cccc}
\toprule
\multicolumn{2}{c}{Model}  &    B$\uparrow$ & R$\uparrow$ & SER$\downarrow$ & OA$\uparrow$    \\
\midrule
\parbox[t]{2mm}{\multirow{8}{*}{\rotatebox[origin=c]{90}{biGRU}}}
 & \textsc{\small{Rnd}} &  50.2&62.0&14.16&--- \\
 & \textsc{\small{Fp}} &  50.5&62.3&16.56&--- \\
 & \textsc{\small{If}} &  51.5&62.7&19.14&--- \\
 & \textsc{\small{If +p}} &  49.6&61.7&12.68&--- \\
 & \textsc{\small{At +BgDP}} &  49.8&60.5&2.64&91.6 \\
 & \textsc{\small{At +NDP}} &  52.9&64.4&1.42&94.3 \\
 & \textsc{\small{At +NDP +p}} &  52.7&63.9&1.24&95.2 \\
 & \textsc{\small{At Oracle}} &  55.7&67.4&1.72&93.2 \\
\midrule
\parbox[t]{2mm}{\multirow{8}{*}{\rotatebox[origin=c]{90}{Transformer}}}
 & \textsc{\small{Rnd}} &  52.6&63.4&8.86&--- \\
 & \textsc{\small{Fp}} &  52.8&63.5&8.36&--- \\
 & \textsc{\small{If}} &  53.3&63.5&6.88&--- \\
 & \textsc{\small{If +p}} &  52.5&63.3&3.80&--- \\
 & \textsc{\small{At +BgDP}} &  48.8&60.3&3.56&80.9 \\
 & \textsc{\small{At +NDP}} &  51.5&63.5&2.68&88.8 \\
 & \textsc{\small{At +NDP +p}} &  51.6&63.4&2.50&88.9 \\
 & \textsc{\small{At Oracle}} &  53.4&66.2&3.64&83.3 \\
\midrule
\parbox[t]{2mm}{\multirow{8}{*}{\rotatebox[origin=c]{90}{BART}}}
 & \textsc{\small{Rnd}} &  45.2&57.4&1.70&--- \\
 & \textsc{\small{Fp}} &  46.2&58.8&1.38&--- \\
 & \textsc{\small{If}} &  44.9&57.0&1.64&--- \\
 & \textsc{\small{If +p}} &  48.4&60.0&1.94&--- \\
 & \textsc{\small{At +BgDP}} &  44.8&56.1&0.64&97.2 \\
 & \textsc{\small{At +NDP}} &  47.5&60.2&0.50&97.8 \\
 & \textsc{\small{At +NDP +p}} &  49.0&61.1&0.98&95.8 \\
 & \textsc{\small{At Oracle}} &  48.7&62.8&1.08&95.8 \\
\bottomrule
\end{tabular}


    \caption{ViGGO validation set (B) \textsc{Bleu}, (R) \textsc{Rouge-L},
    SER, and OA.}
    \label{tab:app.viggo.valid.agg}
\end{table}



\subsection{Validation Results}

Validation set results are shown in \autoref{tab:app.e2e.valid.agg} and 
    \autoref{tab:app.viggo.valid.agg}
for the E2E and ViGGO datasets respectively. Unlike the test results, reported in the main paper and appendix, validation SER and OA are computed automatically and not manually validated. All results are the average of five random
initializations. Also we use the corrected MR produced by our attribute-value
matching rules as input, rather than the original validation set MR.



