We study the degree to which neural sequence-to-sequence models exhibit
fine-grained controllability when performing natural language generation from
a meaning representation.  Using two task-oriented dialogue generation
benchmarks, we systematically compare the effect of four input linearization
strategies on controllability and faithfulness.  Additionally, we evaluate how
a phrase-based data augmentation method can improve performance.  We find that
properly aligning input sequences during training leads to highly controllable
generation, both when training from scratch or when fine-tuning a larger
pre-trained model.  Data augmentation further improves control on difficult,
randomly generated utterance plans.
