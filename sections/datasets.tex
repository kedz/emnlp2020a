\section{Datasets}

We run our experiments on two English language, task-oriented dialogue
datasets, the \href{http://www.macs.hw.ac.uk/InteractionLab/E2E/}{E2E
Challenge} corpus \cite{novikova2017} and the
\href{https://nlds.soe.ucsc.edu/viggo}{ViGGO} corpus \cite{juraska2019}. These
datasets provide MR/utterance pairs from the restaurant and video game
domains, respectively. Examples from the E2E corpus (33,523 train/1,426
dev/630 test) can have up to eight unique attributes.  There is only one
dialogue act for the corpus, \DA{Inform}.  Attribute-values are either binary
or categorical valued.

The ViGGO corpus (5,103 train/246 dev/359 test) contains 14 attribute types
and nine dialogue acts. %(\DA{Request Explanation}, \DA{Recommend}, etc.).  
In
addition to binary and categorical valued attributes, the corpus also features
list-valued attributes (see the \Atr{genres} attribute in
\autoref{fig:examples}) which can have a variable number of values, and an
open-class \Atr{specifier} attribute (see \autoref{app:specifier} for details). 

\subsection{MR/Utterance Alignments} \label{sec:align}

The original datasets do not have alignments between individual
attribute-values and the subspans of the utterances they occur in, which we
need for the \lsshort{At} linearization strategy.  We manually developed a
list of heuristic pattern matching rules (e.g. \textit{not kid-friendly}
$\rightarrow$ \AV{family\_friendly}{no}).  For ViGGO, we started from scratch,
but for E2E we greatly expanded the rule-set created by \citet{dusek2019}.  To
ensure the correctness of the rules, we iteratively added new matching rules,
ran them on the training and validation sets, and verified that they produced
the same MR as was provided in the dataset. This process took one author
roughly two weeks to produce approximately 25,000 and 1,500 rules for the E2E
and ViGGO datasets respectively. Note that the large number of rules is
obtained programmatically, i.e. creating template rules and inserting matching
keywords or phrases (e.g., enumerating variants such as \textit{not
kid-friendly}, \textit{non kid-friendly}, \textit{not family-friendly}, etc.).

In cases where the matching rules produced different MRs than provided in the
original dataset, we manually checked them. In many cases on the E2E dataset
and several times on ViGGO, we found the rule to be correct and the MR to be
incorrect for the given utterance. In those cases, we used the corrected MRs
for training and validation. To maintain comparability to prior work, we
do not modify the test sets in any way. Using the matching rules, we can
determine alignments between the provided MR and the realized utterances.

For most cases, the attribute-values uniquely correspond to an non-overlapping
subspan of the utterance. The \Atr{rating} attribute in the ViGGO dataset,
however, could have multiple reasonable mappings to the utterance, so we treat
it in practice like an addendum to dialogue act, occuring directly after the
dialogue act as part of a ``header'' section in any  MR linearization strategy
(see \autoref{fig:linstrats} where \AV{rating}{N/A} occurs after the dialogue
act regardless of choice of linearization strategy).
