\section{Experiments}

\subsection{Test-Set Evaluation}

In our first experiment, we compare performance of the proposed models and
linearization strategies on the E2E and ViGGO test sets. For the \lsshort{If}
and \lsshort{At+NUP} models we also include variants trained on
the union of original training data and phrase-augmented data (see
\autoref{sec:dataaug}), which we denote \phraseAug.

\paragraph{Evaluation Measures} For automatic quality measures, we report
\bleu~and \rougel~\cite{lin2004} scores.\footnote{We use the
\href{https://github.com/tuetschek/e2e-metrics}{official E2E evaluation
script} to compute these numbers.} Additionally, we use the matching rules to
automatically annotate the attribute-value spans of the model generated
utterances, and then manually verify/correct them. With the attribute-value
annotations in hand we compute the number of missing, wrong, or added
attribute-values for each model. From these counts, we compute the semantic
error rate (SER) \cite{dusek2020} where \[ \textrm{SER} = \frac{\#missing +
\#wrong + \#added}{\#attributes}.\]  On ViGGO, we do not include the
\Atr{rating} attribute in this evaluation since we consider it part of the
dialogue act.  Additionally, for \lsshort{At} variants, we report the order
accuracy (OA) as the percentage of generated utterances that correctly follow
the provided utterance plan. Utterances with wrong or added attribute values
are counted as not following the utterance plan. Additional metrics
and SER error break downs can be found in \autoref{app:exp.results}.

All models are trained five times with different random seeds; we report
the mean of all five runs. We report statistical significance
using Welch's $t$-test \cite{welch1947}, comparing the score distribution of the five runs from the best linearization strategy against all other strategies
at the $0.05$ level.

\paragraph{Baselines} On the ViGGO dataset we compare to the Transformer
baseline of \citet{juraska2019}, which used a beam search of size 10 and
heuristic slot reranker (similar to our matching rules).



On the E2E dataset, we report the results of 
TGen+ \cite{dusek2019}, an
LSTM-based S2S model, which also uses beam search with a matching rule based
reranker to select the most semantically correct utterance and is
trained on a cleaned version of the corpus (similar to our approach).
%, using matching
%rules to correct erroneous MRs in the train data (similar to our approach).
   
\subsection{Random Permutation Stress Test}



Differences between an \lsshort{At} model following a utterance planner model
and the human oracle are often small so we do not learn much about the limits
of controllability of such models, or how they behave in extreme conditions
(i.e. on an arbitrary, random utterance plan, not drawn from the training data
distribution). In order to perform such an experiment we generate random
utterance plans (i.e. permutations of attribute-values) and have the
\lsshort{At} models generate utterances for them, which we evaluate with
respect to SER and OA (we lack ground truth references with which to evaluate
\bleu~or \rougel).  We generate random permutations of size $3,4,\ldots, 8$ on
the E2E dataset, since there are 8 unique attributes on the E2E dataset. For
ViGGO we generate permutations of size $3,4,\ldots,10$ (96\% of the ViGGO
training examples fall within this range). For each size we generated 100
random permutations and all generated plans were given the \DA{Inform}
dialogue act. In addition to running the \lsshort{At} models on these random
permutations, we also compare them to the same model after using the NUP  to
reorder them into an easier\footnote{Easier in the sense that the
\NUP~re-ordering is closer to the training set distribution of \lsshort{At}
utterance plans.} ordering.  Example outputs can be
found in \autoref{app:examples}.  



\subsection{Human Evaluation} In our final experiment, we had human evaluators
rank the 100 outputs of the size 5 random permutations for three \BART~models
on both datasets: (i) \lsshort{At+p} model with \NUP,  (ii) \lsshort{At+p}
model, and (iii) \lsshort{At} model.  The first model, which uses an utterance
planner, is likely to be more natural since it doesn't have to follow the
random order, so it serves as a ceiling.  The second and third models will try
to follow the random permutation ordering, and are more likely to produce
unnatural transitions between awkard sequences of attribute-values.
Differences between these models will allow us to understand how the
phrase-augmented data affects the fluency of the models.  The annotators were
asked to rank outputs by their naturalness/fluency.  Each set was annotated
twice by different annotators so we can compute agreement. More details can be
found in \autoref{app:humaneval}.



