\section{Introduction}

In this work, we study the degree to which sequence-to-sequence (S2S) models
exhibit fine-grained controllability when performing natural language
generation (NLG) from a meaning representation (MR).  In particular, we focus
on an  S2S approach that respects the realization ordering constraints of a
given utterance plan; such a model can generate phrase orderings indicated by
the input plan. 


In non-neural NLG, fine-grained control of the text generation process has
received extensive study under the names \textit{sentence} or
\textit{micro-planning} \cite{reiter2000,walker2001,stone2003}.  Contemporary
practice, however, eschews modeling at this granularity, instead preferring to
train an S2S model to directly map an input MR to a natural language
utterance, with the sentence plan determined implicitly by the patterns
exhibited in the training data and the model's decoder \cite{dusek2020}. 

We argue that robust and fine grained control in an S2S model is desirable
because it enables neural implementations of various psycho-linguistic
theories of discourse (e.g., Centering Theory \cite{grosz1995}, or
Accessibility Theory \cite{ariel2001}).  This could, in turn, encourage the
validation and/or refinement of more psychologically plausible models of
language production.
 
\begin{figure}

{

    \begin{subfigure}{0.5\textwidth}
        \caption{\large~\textbf{E2E MR/Utterance Pair}}
    \end{subfigure}


    \centering
    \begin{subfigure}{0.25\textwidth}
         $\MR{\begin{array}{l}  
            \DA{Inform} \\
            \textrm{\footnotesize food = ``fast food''\textsuperscript{(1)}} \\
            \textrm{\footnotesize price = ``moderate''\textsuperscript{(2)}} \\
            \textrm{\footnotesize name = ``The Punter''\textsuperscript{(3)}}\\
        \end{array}}$
    \end{subfigure}\hfill\begin{subfigure}{0.19\textwidth}
        \textit{\ul{The Punter}\textsuperscript{3} is a 
            \ul{fast food}\textsuperscript{1} joint that is 
            \ul{not too expensive.}\textsuperscript{2}}
    \end{subfigure}
}

~\\



{
    \begin{subfigure}{0.5\textwidth}
        \caption{\large~\textbf{ViGGO MR/Utterance Pair}}
        \label{fig:exviggo}
    \end{subfigure}

    \centering
    \begin{subfigure}{0.25\textwidth}
        $\MR{\begin{array}{l}  
            \DA{Request} \\ 
            \DA{Explanation} \\
            \textrm{\footnotesize genres = [}\\
            \textrm{\footnotesize~~~~``role-playing'',\textsuperscript{(1)}} \\
            \textrm{\footnotesize~~~~``hack-and-slash'',\textsuperscript{(2)}]} \\
            \textrm{\footnotesize ESRB = ``M (Mature)''\textsuperscript{(3)}}\\
            \textrm{\footnotesize rating = ``good''\textsuperscript{(4)}} \\ 
        \end{array}}$

    \end{subfigure}\hfill\begin{subfigure}{0.19\textwidth}
        \textit{What is it about \ul{M rated}\textsuperscript{3} 
            \ul{hack-and-slash}\textsuperscript{2} \ul{RPGs}\textsuperscript{1}
            that \ul{makes you enjoy them?}\textsuperscript{4}}
    \end{subfigure}
}

\caption{(a) Example MR for \DA{Inform} dialogue act (left) and utterance
    (right) pair from the E2E dataset. (b) Example MR for \DA{Request
    Explanation} dialogue act (left) and utterance (right) pair from the ViGGO
    dataset. Superscripts indicate which attribute-values correspond to which
    utterance sub-spans.}
\label{fig:examples}
\end{figure}


In this paper, we study controllability in the context of task-oriented
dialogue generation \cite{mairesse2010,wen2015}, where the input to the NLG
model is an MR consisting of a dialog act (i.e. a communicative goal) such as
to \DA{Request Explanation}, and an unordered set of attribute-value pairs
defining the semantics of the intended utterance (see \autoref{fig:examples}
for examples). 


The NLG model is expected to produce an utterance that adequately and
faithfully communicates the MR.  In the S2S paradigm, the MR must be
``linearized'' (i.e.  represented as a linear sequence of tokens corresponding
to the dialog act and attribute-value pairs) before being presented to the S2S
models's encoder.  We explore several linearization strategies and measure
their effectiveness for controlling phrase order as well as their effect on
the faithfulness (i.e., the semantic correctness of the generated utterance)
of the underlying model.

Of particular note, alignment training (i.e.  at training time, linearizing
the attribute-value pairs according to the order in which they are realized by
their corresponding reference utterance) produces highly controllable  S2S
models.  While we are not the first to observe this (c.f., \citet{nayak2017}),
we study this behavior extensively.  We refer to an ordered sequence of
attribute-value pairs $\attr_1, \attr_2, \ldots, \attr_n$ as an
\textit{utterance plan}, and evaluate models on their ability to follow such
plans given by either another model, a human, or, most difficultly, from
random permutation.

Additionally, we experiment with a data augmentation method, where we create
fragmentary MR/utterance pairs obtained from the constituent phrases of the
original training data.  We find that this data augmentation results in
reduced semantic error rates and increases the ability of a model to follow an
arbitrary utterance plan.

 We summarize our contributions as follows. (1) We show that \textbf{alignment training
produces highly controllable %and faithful 
language generation models},
%in both recurrent and non-recurrent S2S models, 
especially when following a model provided utterance plan. (2) We demonstrate that \textbf{phrase-based
data augmentation improves the robustness of the control} even on arbitrary
and difficult to follow utterance plans. (3) We conclude with a human
evaluation that shows that \textbf{phrase-based data augmentation training
can increase the robustness of control without hurting fluency.}\footnote{
    Code, model outputs, augmented data, and other materials will be 
released on publication.}


