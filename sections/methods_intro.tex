In an MR-to-text task, we are given as input an MR $\mr \in \MRs$ from which
to generate an appropriate natural language utterance $\utt  \in \Utts$, where
\mr~consists of a dialogue act that characterizes the communicative goal of the
utterance and an unordered and variably sized set of attribute-value pairs.
Attributes are either binary or categorical variables (e.g.,
\Atr{family-friendly}: [``yes'', ``no''] or \Atr{food}: [``Chinese'',
``English'', ``French'', $\ldots$]).\footnote{There are also list-valued
attributes but we treat them as individual attribute-value pairs (i.e. in
\autoref{fig:examples}, both \AV{genres}{role-playing} and
\AV{genres}{hack-and-slash} are in $\mr$).}

Let each attribute-value pair \attr~and dialogue act \da~be tokens from a
vocabulary \Attrs, and define the size of an MR, denoted  $\size{\mr}$,  to be
the number of attribute-values $\attr \in \mr$.  A linearization
strategy $\lin : \MRs \rightarrow \Attrs^*$ is a mapping of the dialogue act
and attribute-values in \mr~to an ordered sequence, i.e.  $\lin(\mr) = [\da,
\attr_1, \attr_2, \ldots, \attr_\size{\mr}]$.  Regardless of the choice of
\lin, the first token in $\lin(\mr)$ is always the dialogue act \da.

We experiment with both gated recurrent unit (GRU) \cite{cho2014} and
Transformer \cite{vaswani2017} based S2S model variants to implement a
conditional probability model $\model(\cdot| \lin(\mr); \theta) : \Utts
\rightarrow (0, 1)$ over utterances. The model parameters,
$\theta$ are learned by approximately maximizing the log-likelihood
$\mathcal{L}(\theta)= \sum_{(\mr,\utt)\in \mathcal{D}}  \log
\model(\utt|\lin(\mr); \theta)$ on the training set $\mathcal{D}$.
Additionally, we experiment with a pretrained S2S Transformer, BART
\cite{lewis2020}, with parameters $\theta_0$ fine-tuned on
$\mathcal{L}(\theta_0)$.
