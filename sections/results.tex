\section{Results}


\paragraph{\lsshort{At} models accurately follow utterance plans.} See
\autoref{tab:main.e2e.test} and \autoref{tab:main.viggo.test} for results on
E2E and ViGGO test sets respectively.  
{\color{red}The best non-\Oracle~results are bolded for each model and results
that are not different with statistical significance to the best results
are underlined.}
We see that the \lsshort{At (NUP)}
strategy consistently receives the lowest SER and highest OA, regardless of
architecture, suggesting that alleviating the model's decoder of content
planning is highly beneficial to avoiding errors. The Transformer \lsshort{At} is able to consistently achieve virtually zero SER on E2E.

{\color{red}Generally, the \lsshort{At} models had a smaller variance in test-set
evaluation measures over the five random initializations as compared to the
other strategies. This is reflected in some unusual equivalency classes
by statistical significance. For example, on E2E BART SER, 
\lsshort{At (NUP)+p} is significantly better than \lsshort{At (BgUP)}
(0\% vs 0.2\%), but \textbf{not} \lsshort{If +p} (0\% vs 0.3\%).}


\begin{table}[t!]
    \centering
    \begin{tabular}{ll cccc}
    \toprule
    \multicolumn{2}{c}{Model}&B$\uparrow$&R$\uparrow$&SER$\downarrow$&OA$\uparrow$    \\
    \midrule
    \multicolumn{2}{c}{TGen+} & \multirow{2}{*}{66.0} & \multirow{2}{*}{67.6} &
    \multirow{2}{*}{0.03} & \multirow{2}{*}{---}\\
    \multicolumn{2}{c}{\footnotesize \cite{dusek2019}} & \\
    \midrule
    \parbox[t]{2mm}{\multirow{8}{*}{\rotatebox[origin=c]{90}{biGRU}}}
     & \lsshort{Rnd}  & \textbf{66.8} & 68.3 & 2.64 & --- \\
     & \lsshort{Fp}  & \uline{63.4} & \uline{65.6} & \uline{6.54} & --- \\
     & \lsshort{If}  & 59.2 & 62.7 & 12.64 & --- \\
     & \lsshort{If{\small+p}}  & 65.8 & 68.1 & 0.24 & --- \\
     & \lsshort{At \small{(BgUP)}}  & \uline{66.4} & 68.3 & 0.26 & 98.2 \\
     & \lsshort{At \small{(NUP)}}  & \uline{66.3} & 68.9 & 0.26 & 98.3 \\
     & \lsshort{At \small{(NUP)+p}}  & \uline{66.5} & \textbf{69.1} & \textbf{0.00} & \textbf{100.0} \\
     & \lsshort{At \small{(Oracle)}}  & 69.8 & 77.3 & 0.84 & 94.3 \\
    \midrule
    \parbox[t]{2mm}{\multirow{8}{*}{\rotatebox[origin=c]{90}{Transformer}}}
     & \lsshort{Rnd}  & \textbf{67.4} & 68.2 & \uline{1.06} & --- \\
     & \lsshort{Fp}  & \textbf{67.4} & \uline{68.7} & \uline{3.10} & --- \\
     & \lsshort{If}  & \uline{67.1} & 68.1 & \uline{0.66} & --- \\
     & \lsshort{If\small{+p}}  & \uline{66.8} & 68.3 & \uline{0.28} & --- \\
     & \lsshort{At \small{(BgUP)}}  & \uline{66.8} & 68.4 & \textbf{0.00} & \uline{99.9} \\
     & \lsshort{At \small{(NUP)}}  & \uline{67.0} & \uline{69.0} & \textbf{0.00} & \textbf{100.0} \\
     & \lsshort{At \small{(NUP)+p}}  & \uline{66.7} & \textbf{69.1} & \textbf{0.00} & \textbf{100.0} \\
     & \lsshort{At \small{(Oracle)}}  & 69.3 & 77.0 & 0.76 & 95.0 \\
    \midrule
    \parbox[t]{2mm}{\multirow{8}{*}{\rotatebox[origin=c]{90}{BART}}}
     & \lsshort{Rnd}  & \uline{66.5} & 68.3 & \uline{0.14} & --- \\
     & \lsshort{Fp}  & 65.5 & 67.2 & \uline{0.16} & --- \\
     & \lsshort{If}  & \uline{65.6} & 67.4 & \uline{0.18} & --- \\
     & \lsshort{If\small{+p}}  & \uline{65.9} & 68.2 & \uline{0.30} & --- \\
     & \lsshort{At \small{(BgUP)}}  & \uline{66.2} & 68.7 & 0.20 & 98.6 \\
     & \lsshort{At \small{(NUP)}}  & \textbf{66.6} & \uline{69.2} & 0.20 & 98.6 \\
     & \lsshort{At \small{(NUP)+p}}  & \uline{66.3} & \textbf{69.3} & \textbf{0.00} & \textbf{100.0} \\
     & \lsshort{At \small{(Oracle)}}  & 68.3 & 77.1 & 0.70 & 95.3 \\
    \bottomrule
\end{tabular}



\caption{E2E test set (B) \bleu, (R) \rougel, SER, and OA. All numbers are percents. }
\label{tab:main.e2e.test}
\end{table}

\begin{table}[t!]
    \centering

\begin{tabular}{ll cccc}
\toprule
\multicolumn{2}{c}{Model}&B$\uparrow$&R$\uparrow$&SER$\downarrow$&OA$\uparrow$    \\
\midrule
\multicolumn{2}{c}{Transformer}&
\multirow{2}{*}{52.1}&\multirow{2}{*}{63.8}&\multirow{2}{*}{1.60\tablefootnote{Since their model does not realize \Atr{specifier} attributes, we do not include them in SER calculation. When including them, their model achieves 2.6\% SER.}}& \multirow{2}{*}{---} \\
\multicolumn{2}{c}{\footnotesize \cite{juraska2019}}&\\
\midrule
\parbox[t]{2mm}{\multirow{8}{*}{\rotatebox[origin=c]{90}{biGRU}}}
 &\lsshort{Rnd}  & 50.2 & 61.6 & 12.56 & --- \\
 &\lsshort{Fp}  & 50.2 & 61.0 & 17.12 & --- \\
 & \lsshort{If}  & 50.2 & 61.3 & 19.20 & --- \\
 & \lsshort{If\small{+p}}  & 49.5 & 61.6 & 12.46 & --- \\
 & \lsshort{At \small(BgUP)}  & 48.5 & 58.5 & 3.40 & 89.8 \\
 & \lsshort{At \small{(NUP)}}  & \uline{51.8} & \uline{62.6} & \textbf{1.58} & \uline{93.7} \\
 & \lsshort{At \small{(NUP)+p}}  & \textbf{52.4} & \textbf{62.7} & \uline{1.62} & \textbf{94.3}\\
 & \lsshort{At \small{(Oracle)}}  & 54.1 & 65.5 & 2.42 & 92.2 \\
\midrule
\parbox[t]{2mm}{\multirow{8}{*}{\rotatebox[origin=c]{90}{Transformer}}}
 & \lsshort{Rnd}  & \uline{52.0} & \uline{62.9} & 9.62 & --- \\
 & \lsshort{Fp}  & \textbf{52.6} & \uline{63.0} & 8.70 & --- \\
 & \lsshort{If}  & \uline{52.3} & \uline{62.6} & 7.50 & --- \\
 & \lsshort{If\small{+p}}  & \uline{52.3} & \textbf{63.1} & 4.24 & --- \\
 & \lsshort{At \small{(BgUP)}}  & 48.7 & 59.2 & 4.68 & 79.1 \\
 & \lsshort{At \small{(NUP)}}  & \uline{51.6} & \uline{62.4} & \uline{2.70} & \uline{88.3} \\
 & \lsshort{At \small{(NUP)+p}}  & 51.1 & 62.0 & \textbf{2.28} & \textbf{89.8} \\
 & \lsshort{At \small{(Oracle)}}  & \uline{53.2} & 65.0 & 4.08 & 83.0 \\
\midrule
\parbox[t]{2mm}{\multirow{8}{*}{\rotatebox[origin=c]{90}{BART}}}
 & \textsc{Rnd}  & 43.7 & 55.1 & 1.50 & --- \\
 & \textsc{Fp}  & \uline{47.0} & \uline{58.9} & 1.68 & --- \\
 & \textsc{If}  & 43.1 & 54.4 & 1.86 & --- \\
 & \textsc{If\small{+p}}  & \textbf{49.1} & \textbf{59.7} & \uline{1.78} & --- \\
 & \textsc{At \small{(BgUP)}}  & 43.8 & 54.0 & \uline{0.52} & \textbf{98.3} \\
 & \textsc{At \small{(NUP)}}  & 45.5 & 57.6 & \uline{0.54} & \uline{98.2} \\
 & \textsc{At \small{(NUP)+p}}  & \uline{48.5} & \uline{59.2} & \textbf{0.46} & \uline{98.1} \\
 & \textsc{At \small{(Oracle)}}  & \uline{47.1} & \uline{60.4} & \uline{0.82} & 97.2 \\
\bottomrule
\end{tabular}
\caption{ViGGO test set (B) \bleu, (R) \rougel, SER, and OA. All numbers are percents. }
\label{tab:main.viggo.test}
\end{table}




We also see that fine-tuned BART is able to learn to follow an utterance plan
as well, and is highly competitive with the trained from scratch Transformer
on the E2E dataet (Transformer \textsc{At} (NUP), 0\% SER vs. BART
\textsc{At} (NUP) 0.2\% SER), and surpasses the trained from scratch model in
the small data ViGGO setting (Transformer \textsc{At} (NUP), 2.7\% SER vs.
BART \textsc{At} (NUP) 0.54\% SER). 

\paragraph{Transformer-based models are more faithful than biGRU on
\textsc{Rnd, Fp}, and \textsc{If} linearizations.} On the ViGGO dataset, BART
and Transformer \lsshort{If} achieve 1.86\% and 7.50\% SER respectively, while
the biGRU \lsshort{If} model has 19.20\%. These trends hold for \lsshort{Fp}
and \lsshort{Rnd}, and on the E2E dataset as well. Because there is no
sequential corresondence in the input, it is possible that the recurrence in
the biGRU makes it difficult to ignore spurious input ordering effects.
Additionally, we see that \lsshort{Rnd} does offer some benefits of denoising;
\lsshort{Rnd} models have lower SER than \lsshort{If} models in 3 of 6 cases 
and \lsshort{Fp} models in 5 out of 6 cases.

\paragraph{Model based plans are easier to follow than human reference plans.
} On E2E, there is very little difference in SER between following the
bigram-based utterance planner, \BgUP, or neural utterance planner,
\NUP. This is also true of the ViGGO \BART~models as well.  In the
small data (i.e. ViGGO) setting, trained from scratch \biGRU~and \Transformer~models achieve better SER when following the \NUP.  In most cases,
NUP models have slightly higher \bleu~and \rougel~than
\BgUP, suggesting the NUP model produces utterance plans closer to
the reference orderings. NUP and \textsc{BgUP} models have slightly lower SER
than when following the \Oracle~utterance plans.  This suggests that the models
are producing orders more commonly seen in the training data, similar to how
neural language generators frequently learn the least interesting, lowest
entropy responses \cite{serban2016}.  On the other hand, when given
the \Oracle~orderings, models achieve much higher word overlap with the
reference, e.g. achieving an E2E \textsc{Rouge-L} $\ge 77$.

\begin{table}

    \centering

    \begin{tabular}{llcccc}
\toprule
 & & \multicolumn{2}{c}{E2E} & \multicolumn{2}{c}{ViGGo} \\
\cmidrule(lr){3-4} \cmidrule(lr){5-6}
 \multicolumn{2}{c}{Model} & SER$\downarrow$ & OA$\uparrow$ & SER$\downarrow$ & OA$\uparrow$ \\
\midrule
\multicolumn{2}{l}{biGRU} &
 1.14 & 94.44 & 13.58 & 46.72 \\
 & \small{\textsc{+p }} &
 0.54 & 97.34 & 14.46 & 49.26 \\
 & \small{\textsc{+NUP }} &
 0.22 & 98.72 &  \uline{9.62} & 62.04 \\
 & \small{\textsc{+NUP+p }} &
\textbf{ 0.02} & \textbf{99.86} & \textbf{ 8.98} & \textbf{64.50} \\
\midrule
\multicolumn{2}{l}{Transformer} &
 0.78 & 95.20 & 28.34 & 18.70 \\
 & \small{\textsc{+p }} &
 \uline{0.40} & 98.10 & 25.72 & 18.10 \\
 & \small{\textsc{+NUP }} &
 \uline{0.08} & 99.64 & 24.18 & 31.34 \\
 & \small{\textsc{+NUP+p}} &
\textbf{ 0.02} & \textbf{99.86} & \textbf{21.64} & \textbf{31.86} \\
\midrule
\multicolumn{2}{l}{BART} &
 0.42 & 97.78 &  2.30 & 82.00 \\
 & \small{\textsc{+p }} &
\uline{0.22} & 98.78 &  1.82 & 87.98 \\
 & \small{\textsc{+NUP }} &
 0.64 & 96.52 &  1.34 & 91.40 \\
 & \small{\textsc{+NUP+p}} &
\textbf{ 0.20} & \textbf{99.02} & \textbf{ 0.76} & \textbf{95.32} \\
\bottomrule

    \end{tabular}

\caption{Random permutation stress test of \lsshort{At} models.}
\label{tab:perm}
\end{table}


\paragraph{Phrase-training reduces SER.} We see that phrase data improves SER
in 8 out of 12 cases, with the largest games coming from the biGRU
\lsshort{If} model.  Where the base SER was higher, phrase training has a more
noticeable effect. After phrase training, all E2E models are operating at near
zero SER and almost perfectly following the NUP. Model performance on ViGGO
is more varied, with phrase training slighting hurting the
\lsshort{biGRU At (NUP)} model, but otherwise helping performance.
%while improving for BART
%\lsshort{At (NUP)} models, but improving the Transformer.

\paragraph{Random Permutation Stress Test} Results of the random permutation
experiment are shown in \autoref{tab:perm}.  Overall, all models have an
easier time following the NUP provided utterance plan compared to random
permutations. Phrase training also generally improved SER.  All models perform
quite well on the E2E permutations.  Models had an easier time following the
NUP reordering compared to the random permutations, but with phrase-training,
all E2E models achieve less than 0.6\% SER following random utterance plans.
Starker differences emerge on the ViGGO dataset.  The biGRU +p +NUP model
achieves a 8.98\% SER and only correctly follows the given order 64.5\% of
the time, which is a large decrease in performance compared to that model on
the ViGGO test set (1.62\% SER and 94.3\% OA).

\begin{table}
    \centering
%    \resizebox{0.48\textwidth}{!}{
\begin{tabular}{ll c c c c }
\toprule
 & Model & 1 & 2 & 3 & Avg. \\
\midrule
    \parbox[t]{2mm}{\multirow{3}{*}{\rotatebox[origin=c]{90}{E2E}}}
%E2E & \small{\lsshort{At+NUP+p}} & 123 & 33 & 44 & \textbf{1.61} \\
 & \small{\lsshort{At+NUP+p}} & 61.5 & 16.5 & 22.0 & \textbf{1.61} \\
 %   & \small{\lsshort{At+p}}  & 60 & 88 & 52 & 1.96 \\
    & \small{\lsshort{At+p}}  & 30.0 & 44.0 & 26.0 & 1.96 \\
    %& \small{\lsshort{At}} & 50 & 99 & 51 & 2.01 \\
    & \small{\lsshort{At}} & 25.0 & 49.5 & 25.5 & 2.01 \\
\midrule
    \parbox[t]{2mm}{\multirow{3}{*}{\rotatebox[origin=c]{90}{ViGGO}}}
%ViGGO & \small{\lsshort{At+NUP+p}} & 115 & 55 & 30 & \textbf{1.58} \\
 & \small{\lsshort{At+NUP+p}} & 57.5 & 27.5 & 15.0 & \textbf{1.58} \\
%      & \small{\lsshort{At+p}} & 20 & 59 & 121 & 2.51\\
      & \small{\lsshort{At+p}} & 10.0 & 29.5 & 60.5 & 2.51\\
%      & \small{\lsshort{At}}  &  86 & 92 & 22 & 1.68\\
      & \small{\lsshort{At}}  &  43.0 & 46.0 & 11.0 & 1.68\\
\bottomrule
\end{tabular}
%}
\caption{Human Evaluation results. Table shows the percent of times each model was ranked 1 (best), 2, 3 (worst) in terms of naturalness and average rank.}
\label{tab:human}
\end{table}


\paragraph{Human Evaluation} Results of the human evaluation are shown in
\autoref{tab:human}. We show the number of times each system was ranked 1
(most natural), 2, or 3 (least natural) and the average rank overall.
Overall, we see that model (i) BART \lsshort{At+p} model with the NUP  is
preferred on both datasets, suggesting that the utterance planner is
producing natural orderings of the attribute-values, and the model can
generate reasonable output for it. On the E2E dataset, we also see small
differences in between models (ii) \lsshort{At+p} and (iii) \lsshort{At}
suggesting that when following an arbitrary ordering, the phrase-augmented
model is about as natural as the non-phrase trained model. This is encouraging
as the phrase trained model has lower SER. On the ViGGO dataset we do find
that the phrase trained model is less natural, suggesting that in the small
data setting, phrase-training may hurt fluency when trying to follow a
difficult utterance plan.

For agreement we compute average Kendall's $\tau$ between each pair of
annotators for each dataset. On E2E, we have $\tau=.853$ and ViGGO we have
$\tau=.932$ suggesting very strong agreement.
%($\tau=0$ would indicate rankings
%are random, and $\tau=-1$ would indicate annotators preferred completely
%opposite models). 
